%!TEX root = ./main.tex

% This file is part of the i10 thesis template developed and used by the
% Media Computing Group at RWTH Aachen University.
% The current version of this template can be obtained at
% <http://www.media.informatik.rwth-aachen.de/karrer.html>.



%-----------------------------------------------------------------------------------------------------------------------------------------
% Befehle
% commands
%-----------------------------------------------------------------------------------------------------------------------------------------

%----------------------------------------------------------------------------------
% \myBigFigure	[ LABEL_PREFIX (optional) ]
%				{ FILENAME (without extension) }
%				{ CAPTION TEXT }
%				{ SHORT VERSION OF CAPTION TEXT }
%
%Bild wird in kompletter Breite gesetzt, die Kurzversion der Bildunterschrift erscheint im Abbildungsverzeichnis
%picture using full width of the page, the short caption is what appears in the list of figures index

%----------------------------------------------------------------------------------
% \myFrameBigFigure	[ LABEL_PREFIX (optional) ]
%					{ FILENAME (without extension) }
%					{ CAPTION TEXT }
%					{ SHORT VERSION OF CAPTION TEXT }
%
%Bild wird in kompletter Breite gesetzt und eingerahmt, die Kurzversion der Bildunterschrift erscheint im Abbildungsverzeichnis
%picture with frame using the full width of the page, the short caption is what appears in the list of figures index

%----------------------------------------------------------------------------------
% \myHUGEFigure	[ LABEL_PREFIX (optional) ]
%				{ FILENAME (without extension) }
%				{ CAPTION TEXT }
%				{ SHORT VERSION OF CAPTION TEXT }
%
%Bild wird rotiert und quer in kompletter Breite gesetzt, die Kurzversion der Bildunterschrift erscheint im Abbildungsverzeichnis
%landscape picture using the full width of the rotated page, the short caption is what appears in the list of figures index

%----------------------------------------------------------------------------------
% \myFigure	[ LABEL_PREFIX (optional) ]
%			{ FILENAME (without extension) }
%			{ CAPTION TEXT }
%			{ SHORT VERSION OF CAPTION TEXT }
%
%Bild wird in der Breite der textspalte gesetzt, die Kurzversion der Bildunterschrift erscheint im Abbildungsverzeichnis
%picture using the width of the text column, the short caption is what appears in the list of figures index

%----------------------------------------------------------------------------------
% \myImgRef	[ LABEL_PREFIX (optional) ]
%			{ LABEL OF THE IMAGE }
%
%referenziert das angegebene Bild
%reference to an image

%----------------------------------------------------------------------------------
% \myBigTable	{ YOUR TABULAR DEFINITION }
%			{ CAPTION TEXT }
%			{ TABLE_LABLE }
%
%Tabelle wird in kompletter Breite gesetzt
%table using the full width of the page

%----------------------------------------------------------------------------------
% \myTable	{ YOUR TABULAR DEFINITION }
%			{ CAPTION TEXT }
%			{ TABLE_LABLE }
%
%Tabelle wird in der Breite der Textspalte gesetzt
%table using the width of the text column

%----------------------------------------------------------------------------------
% \myTxtRef	{ LABLE }
%
%Referenz auf Kapitel oder Abschnitte - gibt nummer und namen aus, z.B.: 5.3---"Yaddahyaddah"
%references chapters or sections, outputs number and title, e.g., 5.3---"Yaddahyaddah"

%----------------------------------------------------------------------------------
% \myUnderscore
%
%Setzt einen "sch�nen" Unterstrich f�r URLs
%typesets a 'nice' underscore for URLs

%----------------------------------------------------------------------------------
%\myTilde
%
%Setzt eine "sch�ne" Tilde f�r URLs
%typesets a 'nice' tilde for URLs

%----------------------------------------------------------------------------------
% \myURL	{ TYPESET VERSION OF ANCHOR }
%			{ PRISTINE URL }
%			{ TYPESET VERSION OF URL }
%
%Setzt eine URL
%die typographisch "sch�ne" version erscheint in einer Fu�note,
%im Text erscheint der Ankertext, verlinkt ist die "echte" URL
%typesets a URL
%the typographically correct version appears as a footnote,
%the anchor appears in the text, the link points to the pristine URL

%----------------------------------------------------------------------------------
% \mySimpleURL	{ TYPESET VERSION OF ANCHOR }
%				{ PRISTINE URL }
%
%Setzt eine URL
%die URL erscheint in einer Fu�note,
%im Text erscheint der Ankertext, die URL ist verlinkt
%typesets a URL
%the URL appears as a footnote,
%the anchor appears in the text, the link points to the URL

%----------------------------------------------------------------------------------
% \myProjectURL	{ TYPESET VERSION OF ANCHOR }
%				{ PRISTINE URL INSIDE PROJECT DIRECTORY }
%				{ TYPESET VERSION OF URL INSIDE PROJECT DIRECTORY }
%
%Setzt eine URL innerhalb des Projektverzeichnisses auf "media"
%ACCOUNT muss durch den eigenen Usernamen ersetzt werden
%die typographisch "sch�ne" version erscheint in einer Fu�note,
%im Text erscheint der Ankertext, verlinkt ist die "echte" URL
%typesets a URL inside the project directory on 'media'
%replace ACCOUNT with your username
%the typographically correct version appears as a footnote,
%the anchor appears in the text, the link points to the pristine URL

%----------------------------------------------------------------------------------
% \mnote	{ MARGIN NOTE }
%
%Setzt eine Randnotitz
%puts a comment into the margin in small sans-serif font

%----------------------------------------------------------------------------------
% \todo	{ TODO MARGIN NOTE }
%
%Setzt eine "ToDo"-Randnotitz in rot zur Erinnerung
%puts a 'todo' comment into the margin in red

%----------------------------------------------------------------------------------
% \chapterquote	{ QUOTATION }
%				{ SOURCE }
%
%Setzt ein Zitat zum Einleiten eines Kapitels
%outputs a quote with its source, can be used as an introduction to chapters

%----------------------------------------------------------------------------------
% \myDefBox	{ TERM }
%			{ DEFINITION }
%
%Setzt eine Randnotitz und eine farbige Box (Textspaltenbreite),
%welche einen Begriff und seine Definition enth�lt
%outputs a margin note and a colored box (width of the text column) containing a term and its definition

%----------------------------------------------------------------------------------
% \myBigDefBox	{ TERM }
%				{ DEFINITION }
%
%Setzt eine farbige Box (Seitenbreite), welche einen Begriff und seine Definition enth�lt
%outputs a colored box (width of the page) containing a term and its definition

%----------------------------------------------------------------------------------
% \myDownloadURL	{ TYPESET DOWNLOAD NAME }
%					{ PRISTINE VERSION OF FILENAME }
%					{ TYPESET VERSION OF FILENAME }
%
%Setzt eine farbige Box, welche einen Downloadlink enth�lt
%outputs a colored box containing a download link

%----------------------------------------------------------------------------------
% \emptydoublepage
%
% Leere Doppelseite ohne Kopf- oder Fu�zeile am Ende von Kapiteln
% Clear double page without any header or footer at end of chapters

%----------------------------------------------------------------------------------
% \pagebreak	[ SOME STRANGE LATEX VALUE ]
%
%Eklige pagebreaks f�r den Druck (falls es nicht mehr anders geht)
%pagebreaks for the final print version (last resort weapon against wrong pagebreaks by LaTeX)

%----------------------------------------------------------------------------------
% \TM
%
%Setzt ein (TM) Symbol
%Places a (TM) symbol


%----------------------------------------------------------------------------------
%Packages and parameters
%----------------------------------------------------------------------------------

%Inputencoding f�r den Mac
%inputencoding for the mac
\usepackage[utf8]{inputenc}

%Mathe- und Symbolpakete
%packages for mathematical symbols
\usepackage{latexsym}
\usepackage{amsmath}
\usepackage{amssymb}

%Tabellengestaltung
%table design
\usepackage{booktabs}

%Grafikpaket
%grahics package
\usepackage{color,graphicx}

%relativer Pfad zu den Bildern
%path to your image folder
\graphicspath{{images/}}

%Abs�tze werden nicht eingezogen, sondern vertikal abgesetzt
%do not indent at new paragraphs but add a vertical offset
\usepackage{noindent}

%Palatino+Helvetica statt Computer Modern als standard fonts:
%change standard fonts to Palatino and Helvetica
\usepackage{palatino}

%Bibliographieeinstellungen
%bibliography settings
\usepackage{natbib}
\bibliographystyle{plainnat}

%Zitierbefehle
%citation commands
\newcommand{\fullcite}{\citep} %for "Author [1980]"
\renewcommand{\citeyear}{\citeyearpar} %for "[1980]"

%paket f�r erweiterte kontrollstrukturen
%package for control structures
\usepackage{ifthen}

%marginpar hack --- alle Randnotitzen sollten dann auf der richtigen Seite stehen
%marginpar hack --- moves margin notes to correct position
\usepackage{mparhack}

%lesbare verweise
%make readable references
\usepackage[pdftex,plainpages=false,pdfpagelabels]{hyperref}


%---------------------<Layout in the style of "A Pattern Approach to Interaction Design>---------------------------

% Change page headers and footers:
\usepackage{fancyhdr}
\pagestyle{fancy}
\fancyhf{}
\fancyhead[RE]{\slshape \nouppercase{\leftmark}}    % Even page header: "page   chapter"
\fancyhead[LO]{\slshape \nouppercase{\rightmark}}   % Odd  page header: "section   page"
\fancyhead[RO,LE]{\bfseries \thepage} 
\renewcommand{\headrulewidth}{1pt}    % Underline headers
\renewcommand{\footrulewidth}{0pt}    

\fancypagestyle{plain}{               % No chapter+section on chapter start pages
\fancyhf{}
\fancyhead[RO,LE]{\bfseries \thepage}
\renewcommand{\headrulewidth}{1pt}
\renewcommand{\footrulewidth}{0pt}
}

% Left headings: "1  INTRODUCTION"
\renewcommand{\chaptermark}[1]{%
\markboth{\thechapter\ \ \ \ #1}{}}

% Right headings: "1.1  Basics"
\renewcommand{\sectionmark}[1]{%
\markright{\thesection\ \ \ \ #1}{}}

% some Fancyhdr problem...
\addtolength{\headheight}{2pt} % To avoid overfull vboxes from fancyhdr


%creating a better way to change the layout for the abstract pages
\usepackage{geometry}

\ifthenelse{\lengthtest{\paperheight=250mm}}%
{% -----------------B5 Layout-----------------
% Page layout

%\pdfpageheight250mm
%\pdfpagewidth176mm
\geometry{	b5paper,
			top = 27mm,
			footskip = 10mm,
			inner = 19mm,
			outer = 39mm,
			textheight = 175mm,
			textwidth = 84mm,
			marginparsep = 3mm,
			marginparwidth = 32mm
}
\savegeometry{myText}
% -----------------/ B5 Layout-----------------
}%
{% -----------------A4 Layout-----------------
% Page layout
\geometry{	a4paper,
			twoside,
			includemp,
			includehead,
			top = 30mm,
			headsep = 10mm,
			bindingoffset = 10mm,
			inner = 20mm,
			outer = 40mm,
			bottom = 45mm,
			marginparsep = 10mm,
			marginparwidth = 30mm
}
\savegeometry{myText}
% -----------------/ A4 Layout-----------------
}
% Abstract layout
\geometry{	marginparsep = 0mm,
			marginparwidth = 0mm
}
\savegeometry{myAbstract}
\loadgeometry{myText}

\newlength{\fullwidth} % Width of text plus margin notes
\setlength{\fullwidth}{\textwidth}
\addtolength{\fullwidth}{\marginparsep}
\addtolength{\fullwidth}{\marginparwidth}

\setlength{\headwidth}{\fullwidth} % Header stretches over margin notes


%---------------------</Layout in the style of "A Pattern Approach to Interaction Design>---------------------------


%wird f�r die fl�chendeckende Ausgabe der Titelseite ben�tigt
%needed for the full-face titlepage
\usepackage{eso-pic}

%index verwenden
%make an index
\usepackage{makeidx}
\makeindex

%Index Formatierungshilfen
%formatting helpers for the index
\newcommand{\uu}[1]{\underline{#1}}
\newcommand{\ii}[1]{\textit{#1}}

%neue Definition der Index Umgebung
%redesign of the index
\renewenvironment{theindex}{%
  \vspace*{50pt}%
  {\Huge\bfseries\indexname}\par%
  \vspace*{40pt}%
  \setlength{\parskip}{0pt}%
  \setlength{\parindent}{0pt}%
  \small%
  \renewcommand{\item}{\par{}}%
  \renewcommand{\subitem}{\par\hspace{2em}- }%
}%
{}

%Maximale Gliederungstiefe, die noch ins Inhaltsverzeichnis aufgenommen wird
%maximum depth for the table of contents
\setcounter{tocdepth}{3}

%Vorschlag f�r ein sch�nes Farbschema
%Set of colors which look nice together
\usepackage{color}
\definecolor{orange_light}{rgb}{1,0.8,0.4}
\definecolor{orange_med}{rgb}{0.753,0.62,0.373}
\definecolor{orange_dark}{rgb}{0.506,0.412,0.251}

\definecolor{green_light}{rgb}{0.8,1,0.4}
\definecolor{green_med}{rgb}{0.635,0.745,0.376}
\definecolor{green_dark}{rgb}{0.435,0.498,0.255}

\definecolor{blue_light}{rgb}{0.4,0.8,1}
\definecolor{blue_med}{rgb}{0.365,0.624,0.749}
\definecolor{blue_dark}{rgb}{0.251,0.42,0.502}

\definecolor{pink_light}{rgb}{1,0.435,0.812}
\definecolor{pink_med}{rgb}{0.745,0.38,0.62}
\definecolor{pink_dark}{rgb}{0.498,0.255,0.416}

\definecolor{yellow_light}{rgb}{1,1,0.4}
\definecolor{yellow_med}{rgb}{0.757,0.745,0.373}
\definecolor{yellow_dark}{rgb}{0.506,0.49,0.251}

%blau (f�r URLs)
%blue (for URLs)
\definecolor{blue}{rgb}{0,0,1}

%notwendig f�r die korrekte Erkennung, auf welcher Seite sich eine Abbildung befindet.
%we need this to determine if a figure is on an odd or even page
\usepackage{chngpage}

%Hiermit k�nnen die Abbildungslegenden frei gestaltet werden
%we need this to redesign the captions
\usepackage[font=normalsize,labelfont=bf]{caption}

%Abbildungen kommen auf eine eigene Seite, wenn sie mehr als 85% des Platzes
%auf einer Seite einnehmen
%if a figure takes more than 85% of a page it will be typeset on a separate page
\renewcommand{\floatpagefraction}{0.85}

%Ben�tigt um gro�e Abbildungen gedreht auf eine Seite zu setzen
%we need this to rotate big figures
\usepackage[figuresright]{rotating}

%Verschiedene L�ngenma�e f�r Textboxen
%dimensions for textboxes
\newlength{\myDefBoxWidth}
\setlength{\myDefBoxWidth}{\textwidth}
\addtolength{\myDefBoxWidth}{-4mm}
\newlength{\myBigDefBoxWidth}
\setlength{\myBigDefBoxWidth}{\fullwidth}
\addtolength{\myBigDefBoxWidth}{-4mm}

%Formathilfen f�r MatLab Code (wer's braucht...)
%pre-defined matlab code formats
\usepackage{alltt}
\definecolor{string}{rgb}{0.7,0.0,0.0}
\definecolor{comment}{rgb}{0.13,0.54,0.13}
\definecolor{keyword}{rgb}{0.0,0.0,1.0}



%-----------------------------------------------------------------------------------------------------------------------------------------
% neue Befehle
% new commands
%-----------------------------------------------------------------------------------------------------------------------------------------

%----------------------------------------------------------------------------------
% \myBigFigure	[ LABEL_PREFIX (optional) ]
%				{ FILENAME (without extension) }
%				{ CAPTION TEXT }
%				{ SHORT VERSION OF CAPTION TEXT }
%
%Bild wird in kompletter Breite gesetzt
%picture using full width of the page
\newcommand{\myBigFigure}[4][image]
{%
\begin{figure}[t!bp]%
	\checkoddpage%
	\ifcpoddpage%
		%nothing
	\else
		\hspace{-\marginparsep}\hspace{-\marginparwidth}%
	\fi
	%use minipage to center the label beneath the figure
	\begin{minipage}{\fullwidth}%
		\includegraphics[width= \fullwidth]{#2}%
		\caption[#4]{#3}%
		\label{#1_#2}%
	\end{minipage}%
\end{figure}%
}


%----------------------------------------------------------------------------------
% \myFrameBigFigure	[ LABEL_PREFIX (optional) ]
%					{ FILENAME (without extension) }
%					{ CAPTION TEXT }
%					{ SHORT VERSION OF CAPTION TEXT }
%
%Bild wird in kompletter Breite gesetzt und eingerahmt
%picture with frame using the full width of the page
\newcommand{\myFrameBigFigure}[4][image]
{
\begin{figure}[t!bp]
	\checkoddpage
	\ifcpoddpage
		%nothing
	\else
		\hspace{-\marginparsep}\hspace{-\marginparwidth}
	\fi
	%use minipage to center the label beneath the figure
	\begin{minipage}{\fullwidth}
	\frame{%
		\includegraphics[width= \fullwidth]{#2}%
		}
		\caption[#4]{#3}
		\label{#1_#2}
	\end{minipage}
\end{figure}
}

%----------------------------------------------------------------------------------
% \myHUGEFigure	[ LABEL_PREFIX (optional) ]
%				{ FILENAME (without extension) }
%				{ CAPTION TEXT }
%				{ SHORT VERSION OF CAPTION TEXT }
%
%Bild wird rotiert und quer in kompletter Breite gesetzt
%landscape picture using the full width of the rotated page
\newcommand{\myHugeFigure}[4][image]
{
\begin{sidewaysfigure}[t!bp]
	\checkoddpage
	\ifcpoddpage
		%nothing
		\vspace{\marginparsep}\vspace{\marginparwidth}
	\else
		%nothing
		\vspace{-\marginparsep}\vspace{-\marginparwidth}
	\fi
		\includegraphics[width= \textheight]{#2}
		\caption[#4]{#3}
		\label{#1_#2}
	
\end{sidewaysfigure}
}

%----------------------------------------------------------------------------------
% \myFigure	[ LABEL_PREFIX (optional) ]
%			{ FILENAME (without extension) }
%			{ CAPTION TEXT }
%			{ SHORT VERSION OF CAPTION TEXT }
%
%Bild wird in der Breite der textspalte gesetzt
%picture using the width of the text column
\newcommand{\myFigure}[4][image]%
{%
\begin{figure}[t!bp]%
	\begin{center}%
		\includegraphics[width= \textwidth]{#2}%
		\caption[#4]{#3}
		\label{#1_#2}%
	\end{center}%
\end{figure}%
}%

%----------------------------------------------------------------------------------
% \myImgRef	[ LABEL_PREFIX (optional) ]
%			{ LABEL OF THE IMAGE }
%
%referenziert das angegebene Bild
%reference to an image
\newcommand{\myImgRef}[2][image]%
{%
	\ref{#1_#2}%
}%

%----------------------------------------------------------------------------------
% \myBigTable	{ YOUR TABULAR DEFINITION }
%			{ CAPTION TEXT }
%			{ TABLE_LABLE }
%
%Tabelle wird in kompletter Breite gesetzt
%table using the full width of the page
\newcommand{\myBigTable}[3]%
{%
\begin{table}[htdp]%
	\checkoddpage%
	\ifcpoddpage%
		%nothing
	\else%
		\hspace{-\marginparsep}\hspace{-\marginparwidth}%
	\fi%
	\begin{minipage}{\fullwidth}%
		\begin{center}%
			#1%
			\caption{#2}%
			\label{#3}%
		\end{center}%	
	\end{minipage}%
\end{table}%
}%

%----------------------------------------------------------------------------------
% \myTable	{ YOUR TABULAR DEFINITION }
%			{ CAPTION TEXT }
%			{ TABLE_LABLE }
%
%Tabelle wird in der Breite der Textspalte gesetzt
%table using the width of the text column
\newcommand{\myTable}[3]%
{%
\begin{table}[htdp]%
	\begin{center}%
		#1%
		\caption{#2}%
		\label{#3}%
	\end{center}%	
\end{table}%
}%

%----------------------------------------------------------------------------------
% \myTxtRef	{ LABLE }
%
%Referenz auf Kapitel oder Abschnitte - gibt nummer und namen aus, z.B.: 5.3---"Yaddahyaddah"
%references chapters or sections, outputs number and title, e.g., 5.3---"Yaddahyaddah"
\newcommand{\myTxtRef}[1]
{%
	\ref{#1} ``\nameref{#1}''%
}

%----------------------------------------------------------------------------------
% \myTxtRefPP	{ LABLE }
%
%Referenz auf Kapitel oder Abschnitte - gibt nummer, namen und seiten aus, z.B.: 5.3---"Yaddahyaddah" (p. 45)
%references chapters or sections, outputs number and title, e.g., 5.3---"Yaddahyaddah"
\newcommand{\myTxtRefPP}[1]
{%
	\ref{#1} ``\nameref{#1}'' (p.~\pageref{#1})%
}

%----------------------------------------------------------------------------------
% \myUnderscore
%
%Setzt einen "sch�nen" Unterstrich f�r URLs
%typesets a 'nice' underscore for URLs
\newcommand{\myUnderscore}{$\underline{\hspace{0.5em}}$}

%----------------------------------------------------------------------------------
%\myTilde
%
%Setzt eine "sch�ne" Tilde f�r URLs
%typesets a 'nice' tilde for URLs
\newcommand{\myTilde}{$\sim$}

%----------------------------------------------------------------------------------
% \myURL	{ TYPESET VERSION OF ANCHOR }
%			{ PRISTINE URL }
%			{ TYPESET VERSION OF URL }
%
%Setzt eine URL
%die typographisch "sch�ne" version erscheint in einer Fu�note,
%im Text erscheint der Ankertext, verlinkt ist die "echte" URL
%typesets a URL
%the typographically correct version appears as a footnote,
%the anchor appears in the text, the link points to the pristine URL
\newcommand{\myURL}[3]%
{%
	\textcolor{blue}{%
		\href{#2}{#1}%
	}%
	\footnote{#3}%
}

%----------------------------------------------------------------------------------
% \mySimpleURL	{ TYPESET VERSION OF ANCHOR }
%				{ PRISTINE URL }
%
%Setzt eine URL
%die URL erscheint in einer Fu�note,
%im Text erscheint der Ankertext, die URL ist verlinkt
%typesets a URL
%the URL appears as a footnote,
%the anchor appears in the text, the link points to the URL
\newcommand{\mySimpleURL}[2]%
{%
	\textcolor{blue}{%
		\href{#2}{#1}%
	}%
	\footnote{#2}%
}

%----------------------------------------------------------------------------------
% \myProjectURL	{ TYPESET VERSION OF ANCHOR }
%				{ PRISTINE URL INSIDE PROJECT DIRECTORY }
%				{ TYPESET VERSION OF URL INSIDE PROJECT DIRECTORY }
%
%Setzt eine URL auf hci/public wo die Inhalte des WebServer Ordners auf "oliver" verf�gbar sind
%die typographisch "sch�ne" version erscheint in einer Fu�note,
%im Text erscheint der Ankertext, verlinkt ist die "echte" URL
%typesets a URL to hci/public from where the contents of the WebServer folder from oliver can be accessed
%the typographically correct version appears as a footnote,
%the anchor appears in the text, the link points to the pristine URL
\newcommand{\myProjectURL}[3]%
{%
	\textcolor{blue}{%
		\href{http://hci.rwth-aachen.de/public/#2}{#1}%
	}%
	\footnote{http://hci.rwth-aachen.de/public/#3}%
}

%----------------------------------------------------------------------------------
% \mnote	{ MARGIN NOTE }
%
%Setzt eine Randnotitz
%puts a comment into the margin in small sans-serif font
\newcommand{\mnote}[1]%
{%
	\leavevmode%
	\checkoddpage%
	\ifcpoddpage%
		\marginpar{\raggedright\textsf{{\footnotesize{#1}}}}%
	\else%
		\marginpar{\raggedleft\textsf{{\footnotesize{#1}}}}%
	\fi%
}
	
% leavevmode allows mnotes to be aligned with the first line of a paragraph
% NOTE: you have to put a "%" at the end of the line with the mnote, or you will get an extra blank at the beginning of the paragraph!

%----------------------------------------------------------------------------------
% \todo	{ TODO MARGIN NOTE }
%
%Setzt eine "ToDo"-Randnotitz in rot zur Erinnerung
%puts a 'todo' comment into the margin in red
\definecolor{red}{rgb}{1,0,0}
\newcommand{\todo}[1]{\mnote{\textcolor{red}{ToDo: #1}}}

%----------------------------------------------------------------------------------
% \chapterquote	{ QUOTATION }
%				{ SOURCE }
%
%Setzt ein Zitat zum Einleiten eines Kapitels
%outputs a quote with its source, can be used as an introduction to chapters
\newcommand{\chapterquote}[2]{
\begin{quotation}
    \begin{flushright}
	\noindent\emph{``{#1}''\\[1.5ex]---{#2}}
    \end{flushright}
\end{quotation}
}

%----------------------------------------------------------------------------------
% \myDefBox	{ TERM }
%			{ DEFINITION }
%
%Setzt eine Randnotitz und eine farbige Box (Textspaltenbreite),
%welche einen Begriff und seine Definition enth�lt
%outputs a margin note and a colored box (width of the text column) containing a term and its definition
\newcommand{\myDefBox}[2]
{%
	\setlength{\fboxrule}{1mm}%
	\fcolorbox{orange_med}{orange_light}%
	{%
		\parbox{\myDefBoxWidth}{{\bfseries\scshape#1:}\\#2}%
	}%
	\mnote{Definition:\\\emph{#1}}
}

%----------------------------------------------------------------------------------
% \myBigDefBox	{ TERM }
%				{ DEFINITION }
%
%Setzt eine farbige Box (Seitenbreite), welche einen Begriff und seine Definition enth�lt
%outputs a colored box (width of the page) containing a term and its definition
\newcommand{\myBigDefBox}[2]
{%
	\begin{figure}[h!]
	\setlength{\fboxrule}{1mm}%
	\checkoddpage%
	\ifcpoddpage%
		%nothing
	\else%
		\hspace{-\marginparsep}\hspace{-\marginparwidth}%
	\fi%
	\fcolorbox{orange_med}{orange_light}%
	{%
		\parbox{\myBigDefBoxWidth}{{\bfseries\scshape#1:}\\#2}%
	}%
	\end{figure}
}

%----------------------------------------------------------------------------------
% \myDownloadURL	{ TYPESET DOWNLOAD NAME }
%					{ PRISTINE VERSION OF FILENAME }
%					{ TYPESET VERSION OF FILENAME }
%
%Setzt eine farbige Box, welche einen Downloadlink enth�lt
%outputs a colored box containing a download link
\newcommand{\myDownloadURL}[3]{%
\checkoddpage%
	\ifcpoddpage%
		%nothing
	\else%
		\hspace{-\marginparsep}\hspace{-\marginparwidth}%
	\fi%
\setlength{\fboxrule}{1mm}%
\fcolorbox{green_med}{green_light}{%
\begin{minipage}{\myBigDefBoxWidth}%
\begin{center}%
\myProjectURL{#1}{folder/#2}{folder/#3}%
\end{center}%
\end{minipage}%
}%
}

%----------------------------------------------------------------------------------
% \emptydoublepage
%
% Leere Doppelseite ohne Kopf- oder Fu�zeile am Ende von Kapiteln
% Clear double page without any header or footer at end of chapters
\newcommand{\emptydoublepage}{\clearpage\thispagestyle{empty}\cleardoublepage}

%----------------------------------------------------------------------------------
% \pagebreak	[ SOME STRANGE LATEX VALUE ]
%
%Eklige pagebreaks f�r den Druck (falls es nicht mehr anders geht)
%pagebreaks for the final print version (last resort weapon against wrong pagebreaks by LaTeX)
\newcommand{\PB}[1][3]
{%
	\pagebreak[#1]%
}



%----------------------------------------------------------------------------------
% \TM
%
%Setzt ein (TM) Symbol
%Places a (TM) symbol
\newcommand{\TM}
{%
	\textsuperscript{\texttrademark}%
}
