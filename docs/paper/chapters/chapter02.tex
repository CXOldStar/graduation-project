\chapter{Speaker Recognition Systems}

The process of voice recognition lies on the field of pattern classification, with the speaker and his or her utterance (a speech signal) as inputs for a classifier and a decision as output. This decision may be, given a utterance $\boldsymbol{Y}$ and a set $\boldsymbol{\mathcal{S}} = \{\mathcal{S}_1, ..., \mathcal{S}_S\}$ of known users, ``choose the most likely $\mathcal{S}_i$ from $\boldsymbol{\mathcal{S}}$ who produced $\boldsymbol{Y}$". This is a case of speaker identification and the output is a $\mathcal{S}_i$, for $i = 1, ..., S$. Another type of decision is ``determine if $\mathcal{S}$, who produced $\boldsymbol{Y}$, is the claimed $\mathcal{S}_i$ from $\boldsymbol{\mathcal{S}}$". This is a speaker verification decision, with a binary output (True or False).

[TODO desenvolver PELO MENOS mais UM parágrafo aqui]

\section{Speaker Identification}

\section{Speaker Verification}

\subsection{Likelihood Ratio}

TODO basear-se na seção 2 do artigo ``Speaker Verification Using Adapted Gaussian Mixture Models".