\chapter{Speaker Recognition Systems}

The process of voice recognition lies on the field of pattern classification, with the speaker and his or her utterance (a speech signal) as inputs for a classifier and a decision as output. This decision may be, given a utterance $\boldsymbol{Y}$ produced by a speaker $\mathcal{S}$ and a set $\boldsymbol{\mathcal{S}} = \{\mathcal{S}_1, ..., \mathcal{S}_S\}$ of known users,

\begin{equation}
    \mathcal{S} = \arg\max_i P(\mathcal{S}_i|\boldsymbol{Y}).
    \label{eq:decision_speaker_identification}
\end{equation}

\noindent This is a case of speaker identification and the output is a $\mathcal{S}$ from $\boldsymbol{\mathcal{S}}$. Another type of decision is

\begin{equation}
    \mathcal{S} == \mathcal{S}_i, \text{ if } P(\mathcal{S}_i|\boldsymbol{Y}) \geq \alpha.
    \label{eq:decision_speaker_verification}
\end{equation}

\noindent This is a speaker verification decision, with a binary output (True or False), given a claimed $\mathcal{S}_j$ from $\boldsymbol{\mathcal{S}}$ and a threshold $\alpha$. This chapter (and indirectly the whole document) is about the type of decision given by \equationref{decision_speaker_verification}.

\section{Basic Concepts}

\subsection{Utterance}

\subsection{Dependency x Independency}

\section{Likelihood Ratio Test}

TODO basear-se na seção 2 do artigo ``Speaker Verification Using Adapted Gaussian Mixture Models".

\section{Basic Speaker Verification Architecture}

\subsection{Training Phase}

\subsection{Test Phase}