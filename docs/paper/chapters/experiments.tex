\chapter{Experiments}
\label{ch:experiments}

This chapter details the experiments performed on the system described in the previous chapters, contemplating from the front-end processes until the speaker modeling and the log-likelihood ratio test (see \equationref{score_of_X}). First, a description of the corpus is made. Later, the results are exhibited using the feature extraction process and the GMM-UBM techniques.

\section{Corpus}
\label{sec:corpus}

The database used in the experiments of this work is \emph{The MIT Mobile Device Speaker Verification Corpus} (MIT-MDSCV), \refbib{Woo et. al.}{woo.park.hazen.2006}, a \textbf{corpus} designed to evaluate voice biometric systems of high mobility. All utterances were recorded using mobile devices of different models and manufacturers.

This corpus is composed of three sections. The first, named ``Enroll 1", contains 48 speakers (22 females and 26 males), each with 54 utterances (names of ice cream flavors) of 1.8 seconds average duration, and is used to train the ASR system. The utterances were recorded in three different locations (a quiet office, a mildly noisy hallway, and a busy street intersection) as well as two different microphones (the built-in internal microphone of the handheld device and an external earpiece headset) leading to 6 distinct test conditions. The second section, named ``Enroll 2", is similar to the first with a difference in the order of the spoken utterances, and is used to test the enrolled speakers. The third section, named ``Imposters" is similar to the first two, but with 40 non-enrolled speakers (17 females and 23 males), and is used to test the robustness of the ASR system.

\begin{table}[h]
    \centering
    \begin{tabular}{|l|c|c|ll}
        \cline{1-3}
        \multicolumn{1}{|c|}{\textbf{Section}} & \textbf{Training} & \textbf{Test} &  &  \\ \cline{1-3}
        Enroll 1 & \textbf{X} & \textbf{} &  &  \\ \cline{1-3}
        Enroll 2 & \textbf{} & \textbf{X} &  &  \\ \cline{1-3}
        Imposters & \textbf{} & \textbf{X} &  &  \\ \cline{1-3}
    \end{tabular}
    \caption{Corpus divided in training and test sets.}
    \label{tab:corpus-division}
\end{table}

\section{Coding and Data Preparation}
\label{sec:coding-and-data-preparation}

\section{Experiments and Results}
\label{sec:experiments-and-results}

\subsection{Speaker Identification using GMSM}

\subsection{Speaker Verification using GMSM and UBM}

\subsection{Speaker Verification using AGMSM and UBM}

\subsection{Speaker Identification using FGMSM}

\section{Comparison between GMSM and FGMSM}
