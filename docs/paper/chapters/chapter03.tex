\chapter{Feature Extraction}

The feature extraction process transforms the speech signal in a sequence of vectors
representing the unique characteristics of the speaker's vocal tract. According
to \autocite{wolf.1972}, an ideal characteristic must be:

\begin{itemize}\itemsep0pt
    \item of high inter-speaker and low intra-speaker variability;
    \item robust in the presence of noise and distortion;
    \item frequent and natural in the speech;
    \item easy to measure and extract;
    \item difficult to be artificially produced;
    \item not affected by health issues and long term vocal variations.
\end{itemize}


\section{The Mel Scale}


\section{Mel Frequency Cepstral Coefficient}

TODO referenciar Davis and Mermelstein \autocite{davis.mermelstein.1980},
mostrando que seus experimentos colocam o MFCC como uma técnica de representação
de características melhor que as demais (LFCC, LPC, RC e LPCC).


\section{Energy}