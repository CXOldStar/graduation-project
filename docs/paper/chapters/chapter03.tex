\chapter{Feature Extraction}

As an acoustic wave propagated through space over time, the speech signal is not
appropriate to be used by the speaker verification system. In order to deliver
better outcomes, a good parametric representation must be provided to the system.
This task is performed by the feature extraction process, which transforms a speech
signal into a sequence of characterized measurements (features). The usual objectives
in selecting a representation are (1) to compress the speech data by eliminating
information not pertinent to the phonetic analysis of the data, and (2) to enhance
those aspects of the signal that contribute significantly to the detection of
phonetic differences \autocite{davis.mermelstein.1980}. According to \autocite{wolf.1972}
the ideal features should:

\begin{itemize}\itemsep0pt\parskip0pt
    \item occur naturally and frequently in normal speech;
    \item be easily measurable;
    \item vary as much as possible among speakers, but be as consistent as possible
    for each speaker;
    \item not change overtime or be affected by the speaker's health;
    \item not be affected by reasonable background noise nor depend on specific
    transmission characteristics;
    \item not be modifiable by conscious effort of the speaker, or, at least, be
    unlikely to be affected by attempts to disguise the voice.
\end{itemize}

Features may be categorized based on vocal tract or behavioral aspects, divided
in (1) short-time spectral, (2) spectro-temporal, (3) prosodic and (4) high
level \autocite{pinheiro.2013}. Short-time spectral features usually are calculated
using millisecond length windows and describe the voice spectral envelope, composed
of supralaryngeal properties of the vocal tract, e.g. timbre. Prosodic and
spectro-temporal occur over time, e.g. rhythm and intonation, and high level features
occur during the conversation, e.g. accents.

The parametric representations evaluated in \autocite{davis.mermelstein.1980} may
be divided into those based on the Fourier spectrum, Mel-Frequency Cepstrum
Coefficients (MFCC) and Linear Frequency Cepstrum Coefficients (LFCC), and those
based on the Linear Prediction Spectrum, Linear Prediction Coefficients (LPC),
Reflection Coefficients (RC) and Linear Prediction Cepstrum Coefficients (LPCC).
The better evaluated parametric representation was the MFCC, with minimum and
maximum accuracy of 90.2\% and 99.4\% respectively, leading to its choice as the
parametric representation in this work.


\section{Mel Frequency Cepstral Coefficient}

TODO explicar brevemente o MFCC e quais suas vantagens


\subsection{The Mel Scale}


\subsection{Extraction Process}