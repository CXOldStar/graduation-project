\chapter{Introduction}
\pagenumbering{arabic}

The rise in popularity and naturality of computational systems in the everyday of modern life creates the need for easy and less invasive forms of authentication. While enter a hard to memorize password in a terminal still is the safest approach, voice biometrics presents itself as an alternative with continuing improvement. Also, speech is the most natural way humans have to communicate, being incredibly complex and with numerous specific details related to its producer \autocite{bimbot.et.al.2004}. Therefore, it is expected an increasing usage of vocal interfaces to perform actions such as authenticate in a system, command a machine, and identify who is talking and the content of the conversation.

Over the last decade, voice recognition technology has appeared in many commercial products (e.g. Google Now and Apple Siri) with relatively high popularity. Fingerprint biometrics is a reality for ATMs users, and retina scanners have been commercialized for some decades, then it is natural that the less invasive method of voice recognition be popularized for authentication processes in a near future.

Most of the commercial products based on voice technology are intended to perform \textbf{speech recognition} (determine \emph{what} is being said) instead of \textbf{speaker recognition} (determine \emph{who} is speaking). To achieve this goal, numerous voice processing techniques have become popular, e.g. Natural Language Processing (NLP), Hidden Markov Models (HMM) and Gaussian Mixture Models (GMM). Although all of these are interesting, the subject covered in this paper is a field of speaker recognition and only a small subset of techiniques will be unraveled.

\section{Speaker Recognition}

As stated in \autocite{pinheiro.2013}, speaker recognition may be divided in two fields. The first is \textbf{speaker identification}, aimed to determine the identity of a speaker (using a speech signal) from a non-unitary set of speakers. This task is also named speaker identification in \textbf{closed set}. In the second field, \textbf{speaker verification}, the goal is to determine if a speaker is who he/she claims to be, not an imposter. As the set of imposters is unknown \emph{a priori}, this is an \textbf{open set} problem. A special case is the \textbf{speaker identification in open set}, when an ``imposter class" is added to the system in order to categorize all unmatched speakers found.

Restrictions on the type of text may be used. In \textbf{text-dependent} systems, the content of the speech is relevant to the evaluation, and the training and test utterances must contain the same text (but not the same intonation), e.g. a passphrase. In \textbf{text-independent} systems there is no restriction to the content of both sets. In this case the non-textual characteristics of the user's voice is what have importance to the evaluator. These characteristics are presented in different sentences, usage of different languages and even in gibberish.

This paper is focused in \textbf{text-independent speaker verification}, in other words, the determination of a user's claimed identity by analysis of his/her vocal characteristics with no predefined text to dictate. To achieve that, a GMM adapted from an UBM \autocite{reynolds.quatieri.dunn.2000} is implemented. Also, an adaptation of the technique is proposed and evaluated using the theory of FCM presented in \autocite{gao.zhou.pu.2013}.

\section{Objectives}

The objectives of this study are:

\begin{itemize}\itemsep0pt
    \item Analyse and evaluate the speaker verification system using adapted GMM proposed by \autocite{reynolds.quatieri.dunn.2000};
    \item Propose and evaluate a new method derived from GMM, using the FCM theory proposed by \autocite{gao.zhou.pu.2013};
    \item Conduct experiments and validation of the existent and the proposed methods.
\end{itemize}

\section{Document Structure}

Chapter 2 contains a brief historical context and some basic details about voice recognition, as well as the basic architecture for a speaker verification system. The feature extraction process is explained in chapter 3, from the reasons for its use to the chosen technique (MFCC). In chapter 4 is detailed the GMM and the UBM-GMM. Chapter 5 introduces FCM and the proposed FGMM. Experiments are described in chapter 6, as well as its results. Finally, chapter 7 concludes the study. Furthermore, this work contains an appendix with the most relevant pieces of the source code and some mathematics concepts used.