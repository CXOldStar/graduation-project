\chapter{Introduction}

TODO intro da intro: descrever basicamente porque usar um sistema biométrico, além de iniciar a discussão sobre biometria

\section{Voice Recognition}

TODO explicar com mais detalhes o que é reconhecimento de voz, quais as diferenças entre reconhecimento de speech e de speaker e descrever os tipos de speaker recognition (identification e verification (tema deste trabalho))

\section{Objectives}

The objectives of this study are:

\begin{itemize}\itemsep0pt
    \item Analyse and evaluate the speaker verification system using adapted GMM proposed by Reynolds et al. \autocite{reynolds.quatieri.dunn.2000};
    \item Propose and evaluate a new method derived from GMM, using the FCM theory proposed by Gao et al. \autocite{gao.zhou.pu.2013};
    \item Conduct experiments and validation of the existent and the proposed methods.
\end{itemize}

\section{Document Structure}

Chapter 2 contains a brief historical context and some basic details about voice recognition, as well as the basic architecture for a speaker verification system. The feature extraction process is explained in chapter 3, from the reasons for its use to the chosen technique (MFCC). In chapter 4 is detailed the GMM and the UBM-GMM. Chapter 5 introduces FCM and the proposed FGMM. Experiments are described in chapter 6, as well as its results. Finally, chapter 7 concludes the study. Furthermore, this work contains an appendix with the most relevant pieces of the source code and some mathematics concepts used.